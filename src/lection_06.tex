\section{Лекция 6 21.04.25}
Проблема собственных значений
\\
\(Av = \lambda v, \lambda \ne 0, A\) - матрица \({n \times n}\)
\\
\\
Проблема распадается на:
\begin{enumerate}
    \item Полную: найти все собственные числа/вектора 
    \item Частичную: Нахождение min, max, second max(второй по величине), etc.
\end{enumerate}
А так же необходимо понять насколько найденное решение устойчиво к возмущениям?
\newline

\subsection{Устойчивость}
\subsubsection{Теорема Островского (1920 г.)}  
Пусть \(A_{n \times n}, B_{n \times n}, M = max\{ |a_{ij}|, |b_{ij}|\}, \delta = \frac{1}{nM}\sum_{i,j}|a_{ij} - b_{ij}|\) \\
В таком случае можно пронумеровать собственные числа матриц \(A, B\) так, что \[ (\forall v\ |\lambda_0\langle A \rangle - \lambda_0\langle B \rangle| \leq 2 * (n+1)^2 *M * \delta^{\frac{1}{n}} ) \]

Практический смысл такой: 
\begin{itemize}
    \item При небольших возмущениях: \(\delta \rightarrow 0\)  
    \item Получили что $\delta^{\frac{1}{n}} \rightarrow \infty \Rightarrow 2 * (n+1)^2 *M * \delta^{\frac{1}{n}} \rightarrow \infty $, потенциально неограниченный рост разности модулей собственных чисел, оценка получилась не очень полезной.

\end{itemize}

\subsubsection{Пример}
Из Уилкинсона "Алгебраическая проблема собственных значений": \\
    \[
    A = 
    \begin{pmatrix}
    20 & 20 & 0 &\cdots & \cdots &\cdots & \cdots &0  \\
    0 & 19 & 20 & 0 & \cdots &\cdots  & \cdots & 0\\
    0 & 0 & 18 & 20 & 0 &\cdots & \cdots & 0\\
    0 &\cdots& 0 & 17 & 20 & 0 & \cdots & 0\\
    \vdots & \ddots & \ddots & \ddots & \ddots & \ddots & \ddots & \vdots \\

    0 & \cdots & \cdots & \cdots & \cdots & \cdots & 2 & 20 \\
    0 & \cdots & \cdots & \cdots & \cdots & \cdots & \cdots & 1
    \end{pmatrix}
    \]

        \[
    det(A-\lambda E) = 
    det \begin{pmatrix}
    20-\lambda & 20 & 0 &\cdots & \cdots &\cdots & \cdots &0  \\
    0 & 19-\lambda & 20 & 0 & \cdots &\cdots  & \cdots & 0\\
    0 & 0 & 18-\lambda & 20 & 0 &\cdots & \cdots & 0\\
    0 &\cdots& 0 & 17-\lambda & 20 & 0 & \cdots & 0\\
    \vdots & \ddots & \ddots & \ddots & \ddots & \ddots & \ddots & \vdots \\

    0 & \cdots & \cdots & \cdots & \cdots & \cdots & 2-\lambda & 20\\
    
    0 & \cdots & \cdots & \cdots & \cdots & \cdots & \cdots & 1-\lambda
    \end{pmatrix}
    = \] \\ 
    \[=(1-\lambda)det \begin{pmatrix}
    20-\lambda & 20 & 0 &\cdots & \cdots &\cdots & \cdots &0  \\
    0 & 19-\lambda & 20 & 0 & \cdots &\cdots  & \cdots & 0\\
    0 & 0 & 18-\lambda & 20 & 0 &\cdots & \cdots & 0\\
    0 &\cdots& 0 & 17-\lambda & 20 & 0 & \cdots & 0\\
    \vdots & \ddots & \ddots & \ddots & \ddots & \ddots & \ddots & \vdots \\

    0 & \cdots & \cdots & \cdots & \cdots & \cdots & 2-\lambda & 20\\
    
    \end{pmatrix} =
    \]
    \(
    = (1-\lambda)(2-\lambda)\cdots(20-\lambda) \Rightarrow\) собственные числа это 1, 2, \(\cdots\) 20 \\
    Внесём возмущение: 
        \[
    B = 
    \begin{pmatrix}
    20 & 20 & 0 &\cdots & \cdots &\cdots & \cdots &0  \\
    0 & 19 & 20 & 0 & \cdots &\cdots  & \cdots & 0\\
    0 & 0 & 18 & 20 & 0 &\cdots & \cdots & 0\\
    0 &\cdots& 0 & 17 & 20 & 0 & \cdots & 0\\
    \vdots & \ddots & \ddots & \ddots & \ddots & \ddots & \ddots & \vdots \\

    0 & \cdots & \cdots & \cdots & \cdots & \cdots & 2 & 20 \\
    \epsilon & \cdots & \cdots & \cdots & \cdots & \cdots & \cdots & 1
    \end{pmatrix}
    \] \\
    Тогда:
    \[
    det(B-\lambda E) = 
    det \begin{pmatrix}
    20-\lambda & 20 & 0 &\cdots & \cdots &\cdots & \cdots &0  \\
    0 & 19-\lambda & 20 & 0 & \cdots &\cdots  & \cdots & 0\\
    0 & 0 & 18-\lambda & 20 & 0 &\cdots & \cdots & 0\\
    0 &\cdots& 0 & 17-\lambda & 20 & 0 & \cdots & 0\\
    \vdots & \ddots & \ddots & \ddots & \ddots & \ddots & \ddots & \vdots \\

    0 & \cdots & \cdots & \cdots & \cdots & \cdots & 2-\lambda & 20\\
    
    0 & \cdots & \cdots & \cdots & \cdots & \cdots & \cdots & 1-\lambda
    \end{pmatrix}
    = \]
    \[
    = (20-\lambda) 
    det \begin{pmatrix}
    & 19-\lambda & 20 & 0 & \cdots &\cdots  & \cdots & 0\\
    & 0 & 18-\lambda & 20 & 0 &\cdots & \cdots & 0\\
    &\cdots& 0 & 17-\lambda & 20 & 0 & \cdots & 0\\
    & \ddots & \ddots & \ddots & \ddots & \ddots & \ddots & \vdots \\

    & \cdots & \cdots & \cdots & \cdots & \cdots & 2-\lambda & 20\\
    
    & \cdots & \cdots & \cdots & \cdots & \cdots & \cdots & 1-\lambda
    \end{pmatrix}
    - \] \\
    \[
    -20 * 
    det \begin{pmatrix}
    0 & 20 & 0 &\cdots & \cdots &\cdots & \cdots &0  \\
    0 & 19-\lambda & 20 & 0 & \cdots &\cdots  & \cdots & 0\\
    0 & 0 & 18-\lambda & 20 & 0 &\cdots & \cdots & 0\\
    0 &\cdots& 0 & 17-\lambda & 20 & 0 & \cdots & 0\\
    \vdots & \ddots & \ddots & \ddots & \ddots & \ddots & \ddots & \vdots \\

    0 & \cdots & \cdots & \cdots & \cdots & \cdots & 2-\lambda & 20\\
    
    \epsilon & \cdots & \cdots & \cdots & \cdots & \cdots & \cdots & 1-\lambda
    \end{pmatrix} =\]
    \(= (20- \lambda)(19-\lambda)\cdots(1-\lambda)-20^{19}\epsilon\) \\
Свободный член этого многочлена при \(\epsilon = 4,64*10^{-7}\) будет равен нулю \(\Rightarrow\) появляется собственное число равное 0, меняется характер особенности.
\\ \\
\textbf{Итак получили:} даже небольшое возмущение матрицы(порядка float), приводит к:
    \begin{enumerate}
        \item Изменению характера особенности
        \item Изменение характера собственного числа
    \end{enumerate}
\\

    Можно ли добиться большей устойчивости?

    \subsection{Квадратная диагонализируемая матрица}
    Квадратная матрица называется диагонализируемой, если:
    \begin{enumerate}
        \item В её канонической жордановой форме есть только тривиальные клетки
        \item \(\exists v\) ортогональный, \(D\) - диагональная, что \(Av = vD\), где \(A\) - исходная матрица
    \end{enumerate} \\

    \textbf{Утверждение.} Любая квадратная матрица может быть сколько угодно малыми возмущениями преобразована в диагонализируемую. \\
    
    \textbf{Теорема Бауэра-Файка (1960 г.).} Пусть A - диагонализируемая квадратная матрица, \(\{\lambda_i\}\) - собственные числа, \(M = \Delta A,\tilde{\lambda}\) - собственные числа возмущенной матрицы. \\
    Тогда \(min\{\lambda_i - \tilde{\lambda}\} \leq cond_2(V) ||\Delta A||_2 , V\) - матрица собственных векторов \(A\) \\
    Проблема с $cond_2$: неоднозначно определено, тк $V$ неопределено -> справо неизвестно что. \\
    
    \textbf{Степенной метод} \\
    Идея: $(Av = \lambda v \Rightarrow A(Av) = \lambda(Av) \Rightarrow \lambda \cong \frac{A(Av)}{Av} $ \\ $\Rightarrow $$A(A^2v) = \lambda(A^2v) \Rightarrow \lambda = \frac{A(A^2v)}{A^2v}$ \\
    
    Выберем \(x^{(0)}\) - начальный вектор,\( x^{(k)} = A^{(k)}x{(0)}, x^{(k + 1)} = A^{(k + 1)}x{(0)}, \lambda = \frac{x^{(k+1)}}{x^{(n)}}\) - определено только когда они коллинеарны, отсюда вытекает проблема. \\
    Тогда сам метод: 
    \begin{enumerate}
        \item $x^{(0)}$ - начальный вектор 
        \item $x^{(k)} = A^{(k)}x^{(0)}$
        \item $x^{(k+1)} = A^{(k+1)}x^{(0)}$
        \item $\lambda = \frac{x^{(k+1)}}{x^{(k)}}$
    \end{enumerate}
    \\  
    Но есть проблема: отношение векторов известно только если они коллинеарны.
    \\ Идея решения:
    \begin{itemize}
        \item Пусть i фиксированное, тогда \(\lambda \approx \frac{x_i^{k+1}}{x_i^{k}}\)
        \item Пусть \(L^{(k)}\) - направление. Тогда \(\lambda = \frac{\langle x^{k+1}, L^{n} \rangle}{\langle x^{k}, L^{n} \rangle}, L^{(n)} = x^{(n)}\)
    \end{itemize} \\

    \textbf{Теорема.} Пусть \(A\) - диагонализируемая квадратная матрица с доминирующим собственным числом кратности 1. Пусть \(v_1\) - доминирующий собственный вектор. Если \(x^{(0)} \notin span\{v_2,v_3,\cdots,v_n\}\), то степенной метод сходится к доминирующему собственному вектору из начального приближения \(x^{(0)}\) \\

    \textbf{Доказательство.} \(A\) - диагонализируема \(\Rightarrow \exists V(A = VDV^{-1}), D = diag \{\lambda_1, \lambda_2, \cdots, \lambda_n\}\), не умаляя общности \(\lambda_1\) - доминирующий \\

    \(A^kx^{(0)} = (VDV^{-1})(VDV^{-1})\cdots(VDV^{-1})x^{(0)} = VD^kD^{-1}x^{(0)} = VD^kz 
    = V \begin{pmatrix} 
    \lambda_1^kz_1 \\ 
    \lambda_2^kz_2 \\
    \vdots \\
    \lambda_n^kz_n \\
    \end{pmatrix} = 
    \begin{vmatrix}
        x_0 = Vz \\z_1 \neq 0
    \end{vmatrix} = (\lambda_1^kz_1)V\begin{pmatrix}
        1 \\
        (\frac{\lambda_2}{\lambda_1})^k * z_2/z_1 \\
        \vdots \\
        (\frac{\lambda_n}{\lambda_1})^k * z_n/z_1 \\
    \end{pmatrix} \Rightarrow (\lambda_1^kz_1)V  \begin{pmatrix} 1 \\ 0 \\ \vdots \\ 0 \end{pmatrix} = Cv_1
    \)
    \\
    То есть получили что метод сходится как раз таки к доминирующему собственному вектору.

    Для соблюдения того что $x^{(0)}$, не принидлежит линейной оболочке(в частности не является собственным вектором) на практике используется мультистарт \(x_1^{(0)}, x_2^{(0)} \сdots\) \\

    \textbf{Круги Гершгорина} \\
    Пусть \(v\) - собственный вектор \(A\). Пусть \(l = max_{1 \leq j \leq n}|v_j|, Av = \lambda v, \sum^n_{j=1} a_{lj}v_{j}=\lambda v l \Rightarrow \sum_{j\neq l} a_{l,j}v_{j} = (\lambda - a_{ll})v_l\) \\
    \(|\lambda - a_{ll}||v_l| \leq |\sum_{i \neq j} a_{lj}v_j| \leq \sum_{i \neq j} |a_{lj}||v_j| \leq |v_l|\sum_{j\neq l}|a_{lj}|\), получим \(|\lambda - a_{ll}| \leq \sum_{j \neq l} |a_{lj}|\) \\

    \textbf{Матрица вращений Гивенса.} Вращение вектора в \(R^n\) относительно O \(x_k x_l\).
    \[
    G(k, l, \Theta) = 
    \begin{pmatrix}
    1      & \cdots & \cdots     &\cdots  & \cdots &\cdots & \cdots & \cdots &0  \\
    \vdots & \ddots & \cdots     &\cdots  & \cdots & \cdots &\cdots & \cdots & 0\\
    \vdots & \vdots & cos \Theta & \cdots &  \cdots & sin \Theta & \cdots & \cdots & 0\\
    \vdots & \vdots & \vdots     & 1      & \cdots & \cdots & \cdots & \cdots & 0\\
    \vdots & \vdots & \vdots     & \vdots & \ddots & \cdots & \cdots & \cdots & \vdots \\

    \vdots & \vdots & sin \Theta    & \vdots & \vdots & (-cos \Theta) & \cdots & \cdots & 0 \\
        \vdots & \vdots & \vdots     & \vdots & \vdots & \vdots & 1 & \cdots & 0 \\
    0 & \vdots & \vdots & \vdots & \vdots & \vdots & \vdots & \ddots & 0 \\
        0 & 0 & 0 & 0 &0 & 0 & 0 & \cdots & 1
    \end{pmatrix}
    \] \\
    det(G) = 1 \\
    Основная идея в том чтобы занулить часть компонентов\\
    \textbf{Метод Якоби.} 
    Работает для симметричных матриц. Помним что:
    \newline 1) Матрица собственных чисел 
    \newline 2) В SVD есть симметричная матрица $\begin{pmatrix}
        0 & A^T \\
        A& 0
    \end{pmatrix}$\\
    Называется Якоби потому, что он это использовал в 1846г на матрице 7х7, переоткрыли в 1950-ые.\\

    Идея:
    \newline 1) Уменьшает норму внедиагональной части
    \newline 2) Матрица стремится к диагональной
    \newline 3) У нее собственные числа на диагонали \\
    

    Метод:
    \newline $A^{(0)} = A$ 
    \newline $A^{k+1}=G(p,q,\Theta)^TA^{(k)}G(p,q,\Theta) $\\

    Выбор $\Theta$ проводится так чтобы занулить два внедиагональных элемента\\
    \\ 
    Почему такой метод сходится?\\ \\ 
\textbf{Утверждение.} Фробениусова норма матрицы \(||A||_F = (\sum_{i \neq j}a_{ij}^2)^{\frac{1}{2}}\) не изменяется при левом и правом умножении на ортогональную матрицу. \\
    
    \textbf{Доказательство:} \\
    Пусть \(tr A = \sum_ia_{ij}\\
    ||A||_F = (tr(A^TA))^{\frac{1}{2}}\) \\
    Пусть \(Q\) - ортогональная \(\Rightarrow ||QA||_F=(tr((QA)^TQA)^{\frac{1}{2}} = (tr(A^TA))^{\frac{1}{2}} = ||A||_F\) \\
    \(||AQ||_F = ||(Q^{T}A{T})^{T}||_{F} = ||Q^{T}A^{T}||_{F} = ||A^{T}||_{F} = ||A||_F\) \\
    \\
    Следствие: \(||A^{k+1}||_F=||A^k||_F=\cdots=||A|||_F\) ~--- по построению \(A^{k+1}\) при помощи ортогональных преобразований \\
    \\
    ND(A) = \((\sum_{i \neq j}a_{ij}^2)^{1/2}\) - фробениусова норма внедиагональной части \\
    \(ND(A) = 0 \Leftrightarrow\) матрица диагональная \\

\textbf{Утверждение.} Пусть \(G\) - матрица вращений, такая что \(
    \begin{pmatrix}
      x & x \\ x & x   
    \end{pmatrix} 
    \begin{pmatrix}
        \cdots \\ \cdots
    \end{pmatrix}
    \begin{pmatrix}
        \cdots \\ \cdots
    \end{pmatrix} = 
    \begin{pmatrix}
      x & 0 \\ 0 & x   
    \end{pmatrix}, 
    \) пусть \(B = G^TAG\) и элементы \((p, q), (q, p)\) зануляются, тогда \(ND^2(B) = ND^2(A) - 2a^2pq\) \\

    \textbf{Доказательство.} \(||A||_F = ||B||_F \Rightarrow\)\\
    \(ND^2(B)=||B||_F^2-\sum_ib_{ii}^2=||A||_F^2-(\sum_ia_{ii}^2 - (a_{qq}^2 + a_{pp}^2) + (b_{qq}^2 + b_{pp}^2)) = ND^2(A) - (b_{qq}^2 + b_{pp}^2 -a_{qq}^2 - a_{pp}^2)\) \((*)\)\\
    \(\begin{pmatrix}
        a_{pp} & a_{pq} \\
        a_{qp} & a_{qq}
    \end{pmatrix} \rightarrow \begin{pmatrix}
        b_{pp} & 0 \\
        0 & b_{qq}
    \end{pmatrix}  \Rightarrow b_{pp}^2 + b_{qq^2}=a^2_{pp} + a^2_{qq} + 2a_{pq}^2\) ~--- так как фробениусова норма сохраняется при ортогональном преобразовании. Подставим это в \((*)\) и получим условие. \qed\\
    \\
    \(\Rightarrow ||\cdot||_{F}\) внедиагональной части монотонно убывает при каждом преобразовании (вращении)
    \\

\textbf{Когда останавливаться?}
    \begin{itemize}
        \item \(ND(A) < \epsilon \)
        \item \(A^{(k)}=(a_{ij}), \delta = max \sum_{i\neq j}|a_{ij}|, \lambda_i \in [a_{ii}-\Delta, a_{ii} + \Delta]\)
    \end{itemize} \\

\textbf{Как выбрать очередной обнуляемый элемент (p, q)?}
    \begin{itemize}
        \item \(a_{pq}\) - max. Локально оптимально, но поиск за \(\rightarrow O(n^2)\) ~--- не лучший вариант
        \item Барьерный подход - зануляем все, что больше выбранных \(b_{1}, b_{2},..\), пока не достигнем критерия остановки.
    \end{itemize} \\
    
\textbf{Замечания}
    \begin{itemize}
        \item Существует метод Гивенса ~--- оптимизированный метод Якоби ~--- позволяет не оживать однажды занулённым элементам
        \item Метод Якоби хорошо поддаётся распараллеливанию, что может несколько ускорить вычисления на ЭВМ.
    \end{itemize} \\
    \\

\textbf{QR-алгоритм}
    \begin{itemize}
        \item Представлен в 1960г. В.Н.Кублановской
        \item Идея: представим \(A\) в виде \(A = QR\), где \(R\) ~--- верхнетреугольная, а \(Q\) ~--- ортогональная
        \item Для этого положим \(A^{(0)} := A\) и \(A^{(k+1)} = R^{(k)}Q^{(k)}\)
        \item Заметим, что \(A^{(k+1)} = R^{(k)}Q^{(k)} = (Q^{(k)})^{-1}Q^{(k)}R^{(k)}Q^{(k)} = (Q^{(k)})^{-1}A^{(k)}Q^{(k)}\) ~--- то есть такие преобразования не меняют собственные числа матрицы \(A\)
        \item Сходится к диагональной матрице с собственными числами, но обосновать эту сходимость непросто
    \end{itemize}
    